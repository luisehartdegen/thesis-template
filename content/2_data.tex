\chapter{Beschreibung}\label{ch:data}

\section{Problembeschreibung}
% *worst version possible*

Der Arbeitsprozess ein 3D-Modell zu erstellen und mittels eines 3D-Druckers zu fertigen, beinhaltet einige Arbeitsschritte, die sich h\"aufig wiederholen. 
Der \"ubliche Ablauf ist dabei, die 3D-Datei in einer Software wie Autodesk Fusion zu erstellen und als STL- oder 3MF-Datei zu exportieren.
Dann wird diese Datei in eine Slicer-Software importiert, wo die Druckparameter festgelegt werden, bspw.\ die Filamente und die Fl\"ache des Objekts, die auf dem Druckbett aufliegt. 
Da der Prozess oft iterativ abl\"auft, m\"ussen diese Druckparameter nach jedem Import des leicht ver\"anderten Modells erneut festgelegt werden, was zeitaufwendig und umst\"andlich ist.

betroffene Personen(kreise), Szenarien\dots

\section{Forschungsfrage}
Im Rahmen der Bachelorarbeit soll untersucht werden, ob es machbar ist, beim Exportieren der 3MF-Datei sogenannte Print Hints in der Datei zu hinterlegen und diese im Orca Slicer auszulesen und automatisch zu verarbeiten, um die oben beschriebenen Arbeitsabläufe effizienter zu gestalten.

\section{Lösungsansatz}
Der aktuell angedachte Ansatz umfasst mehrere Komponenten. 
In erster Linie soll der Import im Orca Slicer dahingehend \"uberarbeitet werden, dass die Materialien aus der 3MF-Datei ausgelesen werden und automatisch bestimmte Filament-Presets bestimmten Farben zugeordnet werden. % genauer ausf\"uhren + how? plugin? erweiterung des codes? 
Des Weiteren soll es m\"oglich sein, die Druckfl\"ache direkt in Autodesk Fusion zu definieren. 
Daf\"ur ist ggf.\ ein Plugin f\"ur Autodesk Fusion n\"otig sowie eine \"Anderung des Dateiformats, da derartige Flags bisher nicht vorgesehen sind.

wie soll das Problem gelöst werden? geänderte Arbeitsabläufe, Beschreibung technischer Systeme

\section{Forschungsmethodik}
Da es sich hier um eine Machbarkeitsstudie handelt, wird die Methodik in erster Linie Prototypen umfassen. 
Der Wichtigste ist dabei der ge\"anderte Slicer oder alternativ ein Plugin f\"ur diesen. 
Sekund\"ar kommt das ge\"anderte Dateiformat dazu sowie ggf.\ ein Plugin f\"ur Autodesk Fusion.
Optional k\"onnen noch Nutzerbefragungen durchgef\"uhrt werden, um die Usability zu pr\"ufen.

wie wird die Forschungsfrage beantwortet? Messungen / Benchmarks, Nutzerbefragungen, analytische Methoden, Prototyp\dots

\section{Forschungsstand}
Bisherige Erkenntnisse: 
\begin{itemize}
    \item die Materialien werden als Hexcode-Farben im 3MF-Dateiformat gespeichert
    \item das 3MF-Dateiformat speichert die 3D-Modelle in gut lesbarem und leicht zu ver\"anderndem XML-Format ab 
    \item 3MF-Dateien k\"onnen problemlos im Orca Slicer eingelesen werden, wenn manuell Flags hinzugef\"ugt wurden
\end{itemize}

bisherige Arbeiten, wie ist der aktuelle Stand; welche relevanten Ver\"offentlichungen gibt es schon? 