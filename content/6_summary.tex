\chapter{Fazit}\label{ch:summary}

Zusammenfassend kann man sagen, dass Kubernetes zu Recht als De-facto-Standard im Bereich Containerorchestrierung gilt und eine enorme Bandbreite an Funktionen bietet. 
Jedoch geht diese Vielseitigkeit auch mit einer gewissen Einstiegshürde einher. 
Die Einrichtung und Verwaltung eines Kubernetes-Clusters erfordert tiefgehendes Wissen und ist insbesondere für kleinere Projekte mit zusätzlichem Aufwand verbunden.

Wer sich beruflich im Bereich der Containerorchestrierung weiterentwickeln möchte, sollte sich definitiv mit Kubernetes auseinandersetzen. 
Durch seine breite Verwendung in der Industrie und den hohen Funktionsumfang ist es eine essenzielle Technologie für DevOps- und Cloud-native Umgebungen.

Für viele Projekte, insbesondere im universitären oder experimentellen Umfeld, ist jedoch nicht immer eine so komplexe Lösung erforderlich. 
In Szenarien, in denen die Containerorchestrierung nur ein Aspekt unter vielen ist, bietet Docker Swarm eine deutlich einfachere Alternative mit niedrigeren Einstiegshürden, insbesondere wenn die beteiligten Personen sich bereits in Ans\"atzen mit Docker auseinander gesetzt haben. 
Auch für produktive Umgebungen, in denen eine schlanke und unkomplizierte Verwaltung ausreicht, kann Docker Swarm eine sinnvolle Wahl sein.

Letztlich hängt die Wahl des richtigen Orchestrierungswerkzeugs von den individuellen Anforderungen des Projekts ab. 
Während Kubernetes für großflächige, skalierbare Systeme die erste Wahl ist, kann Docker Swarm eine pragmatische Lösung für weniger komplexe Anwendungsfälle und Lernumgebungen darstellen.